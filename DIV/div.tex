%\documentstyle[11pt,a4]{article}
%\documentclass[a4paper]{article}
\documentclass[11pt]{article}
% Seems like it does not support 9pt and less. Anyways I should stick to 10pt.
%\documentclass[a4paper, 9pt]{article}
%\topmargin-2.0cm

\usepackage{fancyhdr}
\usepackage{pagecounting}
\usepackage[dvips]{color}
\usepackage{wrapfig}
\usepackage{graphicx}
\usepackage{hyperref}

\def\ie{{\em i.e.,}}
\def\eg{{\em e.g.,}}
\newcommand{\tabref}[1]{Table~\ref{#1}}
\newcommand{\figref}[1]{Figure~\ref{#1}}
\newcommand{\secref}[1]{Section~\ref{#1}}
\newcommand{\comment}[1]{}

\advance\oddsidemargin-0.6in
%\advance\evensidemargin-1.5cm
\textheight9.2in
\textwidth6in
\newcommand\bb[1]{\mbox{\em #1}}
\def\baselinestretch{1.05}
%\pagestyle{empty}

\newcommand{\hsp}{\hspace*{\parindent}}
\definecolor{gray}{rgb}{0.4,0.4,0.4}
%\definecolor{gray}{rgb}{1.0,1.0,1.0}


\begin{document}
\thispagestyle{fancy}
%\pagenumbering{gobble}
%\fancyhead[location]{text} 
% Leave Left and Right Header empty.
\lhead{}
\rhead{}
%\rhead{\thepage}
\renewcommand{\headrulewidth}{0pt} 
\renewcommand{\footrulewidth}{0pt} 
\fancyfoot[C]{\footnotesize \textcolor{gray}{\href{https://norouzi.github.io/}{https://norouzi.github.io/}}} 

%\pagestyle{myheadings}
%\markboth{Sundar Iyer}{Sundar Iyer}

\pagestyle{fancy}
\lhead{\textcolor{gray}{\it Mohammad Norouzi}}
\rhead{\textcolor{gray}{\thepage/\totalpages{}}}
%\rhead{\thepage}
%\renewcommand{\headrulewidth}{0pt} 
%\renewcommand{\footrulewidth}{0pt} 
%\fancyfoot[C]{\footnotesize http://www.stanford.edu/$\sim$sundaes/application} 
%\ref{TotPages}

% This kind of makes 10pt to 9 pt.
%\begin{small}

%\vspace*{0.1cm}
\begin{center}
{\LARGE \bf Diversity Statement}\\
\vspace*{0.1cm}
{\normalsize Mohammad Norouzi (mnorouzi@google.com)}
\vspace*{0.2cm}
\end{center}
%\vspace*{0.2cm}

%\begin{document}
%\centerline {\Large \bf Research Statement for Mohammad Norouzi}
%\vspace{0.5cm}

% Write about research interests...
%\footnotemark
%\footnotetext{Check This}

The Machine Learning research community deeply suffers from an
underrepresentation of women, people of color, sexual minorities, and
people with disabilities. If we believe that the future of technology
hinges on {\em Artificial Intelligence (AI)}, we should get everyone
involved in the development of the foundations of AI, so that the
future technology engages with the needs and interests of a wider
population. If I am given the opportunity to advise graduate students
at your esteemed university, I will make it my priority to assemble a
research group that includes women, people of color, and other
minorities. I will actively encourage and empower my undergraduate
students of all backgrounds to pursue a career in research, especially
in AI.

I have long been committed to promoting a culture of inclusion and
diversity.  During graduate school, one of my colleagues underwent
male to female transition, and I tried my best to support them during
the difficult process. In my current role as a Research Scientist at
Google Brain, I have mentored 3 women (Sara Sabour, Yijie Guo, and
Azalia Mirhosseini) who led $3$ of my recent projects resulting in two
papers at ICLR $2019$ and one at ICML $2017$. I continue to work with
Sara and Yijie on other projects. I have collaborated with researchers
with different ethnicities, sexual orientations, and religious
beliefs, and I believe that if people with different backgrounds are
provided with equal opportunities and proper research mentorship, they
can equally excel in research. I have been involved in recruiting for
the Google Brain Toronto team, and I am proud to share that out of a
group of $9$ full-time researchers, we have $3$ women. This ratio can
certainly improve and we hope to involve even more women and
minorities.

Being a minority myself, I have witnessed firsthand that minorities
often need to work harder to prove themselves. I have taken courses on
{\em unconscious bias} at Google and I have learned about different
types of bias and the mechanisms of stereotyping. I am committed to
mitigating the negative impact of stereotype bias and to helping my
research group and community to foster a fair and inclusive culture.

\comment{
The Machine Learning research community deeply suffers from an under
representation of women, people of color, and other minorities. If we
believe that the future of technology hinges on {\em Artificial
  Intelligence (AI)}, we should get everyone involved in the
development of the foundations of AI, so that the future technology
engages with the needs and interests of a wider population. If I am
given the opportunity to advise graduate students at your esteemed
university, I will make it my priority to assemble a research group
that includes women, people of color, and other minorities. I will
actively encourage and empower my undergraduate students of all
backgrounds to pursue a carrier in research, especially in AI.

During graduate school, one of my colleagues underwent male to female
transition, and I tried my best to support them during the difficult
process. Having witnessed the complexity of this matter, I am
committed to make it easier for my students and colleagues to express
their gender in the way that provides them with long-term tranquility.

I have long been committed to promoting a culture of inclusion and
diversity. In my current role as a Research Scientist at Google Brain,
I have mentored $3$ women (``Sara Sabour'', ``Yijie Guo'', and
``Azalia Mirhosseini'') who led $3$ of my recent projects resulting in
two papers at ICLR $2019$ and one at ICML $2017$. I have collaborated
with different researchers with different ethnicities, sexual
orientations, and religious beliefs, and I have come to the conclusion
for myself that if people with different backgrounds are provided with
equal opportunities and research mentorship, they can equally excel in
research. I have been involved in recruiting for the Google Brain
Toronto team, and I am happy to share that out of a group of $9$ full
time researchers we have $3$ women. This ratio is not ideal yet, but
at least we have thought about the under representation of women in AI
research and have taken some steps to help address this issue.

Being a minority myself, I have witnessed first hand that minorities
often need to work harder to prove themselves. I have taken courses on
{\em unconscious bias} at Google and I have learned about different
types of bias and the mechanisms of stereotyping. I am deeply
committed to mitigate the unintended negative impact of stereotype
bias and help my research group and community to foster a fair and
inclusive culture.
}

\vspace{0.5cm}

\end{document}

